
% Default to the notebook output style

    


% Inherit from the specified cell style.




    
\documentclass[11pt]{article}

    
    
    \usepackage[T1]{fontenc}
    % Nicer default font (+ math font) than Computer Modern for most use cases
    \usepackage{mathpazo}

    % Basic figure setup, for now with no caption control since it's done
    % automatically by Pandoc (which extracts ![](path) syntax from Markdown).
    \usepackage{graphicx}
    % We will generate all images so they have a width \maxwidth. This means
    % that they will get their normal width if they fit onto the page, but
    % are scaled down if they would overflow the margins.
    \makeatletter
    \def\maxwidth{\ifdim\Gin@nat@width>\linewidth\linewidth
    \else\Gin@nat@width\fi}
    \makeatother
    \let\Oldincludegraphics\includegraphics
    % Set max figure width to be 80% of text width, for now hardcoded.
    \renewcommand{\includegraphics}[1]{\Oldincludegraphics[width=.8\maxwidth]{#1}}
    % Ensure that by default, figures have no caption (until we provide a
    % proper Figure object with a Caption API and a way to capture that
    % in the conversion process - todo).
    \usepackage{caption}
    \DeclareCaptionLabelFormat{nolabel}{}
    \captionsetup{labelformat=nolabel}

    \usepackage{adjustbox} % Used to constrain images to a maximum size 
    \usepackage{xcolor} % Allow colors to be defined
    \usepackage{enumerate} % Needed for markdown enumerations to work
    \usepackage{geometry} % Used to adjust the document margins
    \usepackage{amsmath} % Equations
    \usepackage{amssymb} % Equations
    \usepackage{textcomp} % defines textquotesingle
    % Hack from http://tex.stackexchange.com/a/47451/13684:
    \AtBeginDocument{%
        \def\PYZsq{\textquotesingle}% Upright quotes in Pygmentized code
    }
    \usepackage{upquote} % Upright quotes for verbatim code
    \usepackage{eurosym} % defines \euro
    \usepackage[mathletters]{ucs} % Extended unicode (utf-8) support
    \usepackage[utf8x]{inputenc} % Allow utf-8 characters in the tex document
    \usepackage{fancyvrb} % verbatim replacement that allows latex
    \usepackage{grffile} % extends the file name processing of package graphics 
                         % to support a larger range 
    % The hyperref package gives us a pdf with properly built
    % internal navigation ('pdf bookmarks' for the table of contents,
    % internal cross-reference links, web links for URLs, etc.)
    \usepackage{hyperref}
    \usepackage{longtable} % longtable support required by pandoc >1.10
    \usepackage{booktabs}  % table support for pandoc > 1.12.2
    \usepackage[inline]{enumitem} % IRkernel/repr support (it uses the enumerate* environment)
    \usepackage[normalem]{ulem} % ulem is needed to support strikethroughs (\sout)
                                % normalem makes italics be italics, not underlines
    

    
    
    % Colors for the hyperref package
    \definecolor{urlcolor}{rgb}{0,.145,.698}
    \definecolor{linkcolor}{rgb}{.71,0.21,0.01}
    \definecolor{citecolor}{rgb}{.12,.54,.11}

    % ANSI colors
    \definecolor{ansi-black}{HTML}{3E424D}
    \definecolor{ansi-black-intense}{HTML}{282C36}
    \definecolor{ansi-red}{HTML}{E75C58}
    \definecolor{ansi-red-intense}{HTML}{B22B31}
    \definecolor{ansi-green}{HTML}{00A250}
    \definecolor{ansi-green-intense}{HTML}{007427}
    \definecolor{ansi-yellow}{HTML}{DDB62B}
    \definecolor{ansi-yellow-intense}{HTML}{B27D12}
    \definecolor{ansi-blue}{HTML}{208FFB}
    \definecolor{ansi-blue-intense}{HTML}{0065CA}
    \definecolor{ansi-magenta}{HTML}{D160C4}
    \definecolor{ansi-magenta-intense}{HTML}{A03196}
    \definecolor{ansi-cyan}{HTML}{60C6C8}
    \definecolor{ansi-cyan-intense}{HTML}{258F8F}
    \definecolor{ansi-white}{HTML}{C5C1B4}
    \definecolor{ansi-white-intense}{HTML}{A1A6B2}

    % commands and environments needed by pandoc snippets
    % extracted from the output of `pandoc -s`
    \providecommand{\tightlist}{%
      \setlength{\itemsep}{0pt}\setlength{\parskip}{0pt}}
    \DefineVerbatimEnvironment{Highlighting}{Verbatim}{commandchars=\\\{\}}
    % Add ',fontsize=\small' for more characters per line
    \newenvironment{Shaded}{}{}
    \newcommand{\KeywordTok}[1]{\textcolor[rgb]{0.00,0.44,0.13}{\textbf{{#1}}}}
    \newcommand{\DataTypeTok}[1]{\textcolor[rgb]{0.56,0.13,0.00}{{#1}}}
    \newcommand{\DecValTok}[1]{\textcolor[rgb]{0.25,0.63,0.44}{{#1}}}
    \newcommand{\BaseNTok}[1]{\textcolor[rgb]{0.25,0.63,0.44}{{#1}}}
    \newcommand{\FloatTok}[1]{\textcolor[rgb]{0.25,0.63,0.44}{{#1}}}
    \newcommand{\CharTok}[1]{\textcolor[rgb]{0.25,0.44,0.63}{{#1}}}
    \newcommand{\StringTok}[1]{\textcolor[rgb]{0.25,0.44,0.63}{{#1}}}
    \newcommand{\CommentTok}[1]{\textcolor[rgb]{0.38,0.63,0.69}{\textit{{#1}}}}
    \newcommand{\OtherTok}[1]{\textcolor[rgb]{0.00,0.44,0.13}{{#1}}}
    \newcommand{\AlertTok}[1]{\textcolor[rgb]{1.00,0.00,0.00}{\textbf{{#1}}}}
    \newcommand{\FunctionTok}[1]{\textcolor[rgb]{0.02,0.16,0.49}{{#1}}}
    \newcommand{\RegionMarkerTok}[1]{{#1}}
    \newcommand{\ErrorTok}[1]{\textcolor[rgb]{1.00,0.00,0.00}{\textbf{{#1}}}}
    \newcommand{\NormalTok}[1]{{#1}}
    
    % Additional commands for more recent versions of Pandoc
    \newcommand{\ConstantTok}[1]{\textcolor[rgb]{0.53,0.00,0.00}{{#1}}}
    \newcommand{\SpecialCharTok}[1]{\textcolor[rgb]{0.25,0.44,0.63}{{#1}}}
    \newcommand{\VerbatimStringTok}[1]{\textcolor[rgb]{0.25,0.44,0.63}{{#1}}}
    \newcommand{\SpecialStringTok}[1]{\textcolor[rgb]{0.73,0.40,0.53}{{#1}}}
    \newcommand{\ImportTok}[1]{{#1}}
    \newcommand{\DocumentationTok}[1]{\textcolor[rgb]{0.73,0.13,0.13}{\textit{{#1}}}}
    \newcommand{\AnnotationTok}[1]{\textcolor[rgb]{0.38,0.63,0.69}{\textbf{\textit{{#1}}}}}
    \newcommand{\CommentVarTok}[1]{\textcolor[rgb]{0.38,0.63,0.69}{\textbf{\textit{{#1}}}}}
    \newcommand{\VariableTok}[1]{\textcolor[rgb]{0.10,0.09,0.49}{{#1}}}
    \newcommand{\ControlFlowTok}[1]{\textcolor[rgb]{0.00,0.44,0.13}{\textbf{{#1}}}}
    \newcommand{\OperatorTok}[1]{\textcolor[rgb]{0.40,0.40,0.40}{{#1}}}
    \newcommand{\BuiltInTok}[1]{{#1}}
    \newcommand{\ExtensionTok}[1]{{#1}}
    \newcommand{\PreprocessorTok}[1]{\textcolor[rgb]{0.74,0.48,0.00}{{#1}}}
    \newcommand{\AttributeTok}[1]{\textcolor[rgb]{0.49,0.56,0.16}{{#1}}}
    \newcommand{\InformationTok}[1]{\textcolor[rgb]{0.38,0.63,0.69}{\textbf{\textit{{#1}}}}}
    \newcommand{\WarningTok}[1]{\textcolor[rgb]{0.38,0.63,0.69}{\textbf{\textit{{#1}}}}}
    
    
    % Define a nice break command that doesn't care if a line doesn't already
    % exist.
    \def\br{\hspace*{\fill} \\* }
    % Math Jax compatability definitions
    \def\gt{>}
    \def\lt{<}
    % Document parameters
    \title{????\_????}
    
    
    

    % Pygments definitions
    
\makeatletter
\def\PY@reset{\let\PY@it=\relax \let\PY@bf=\relax%
    \let\PY@ul=\relax \let\PY@tc=\relax%
    \let\PY@bc=\relax \let\PY@ff=\relax}
\def\PY@tok#1{\csname PY@tok@#1\endcsname}
\def\PY@toks#1+{\ifx\relax#1\empty\else%
    \PY@tok{#1}\expandafter\PY@toks\fi}
\def\PY@do#1{\PY@bc{\PY@tc{\PY@ul{%
    \PY@it{\PY@bf{\PY@ff{#1}}}}}}}
\def\PY#1#2{\PY@reset\PY@toks#1+\relax+\PY@do{#2}}

\expandafter\def\csname PY@tok@w\endcsname{\def\PY@tc##1{\textcolor[rgb]{0.73,0.73,0.73}{##1}}}
\expandafter\def\csname PY@tok@c\endcsname{\let\PY@it=\textit\def\PY@tc##1{\textcolor[rgb]{0.25,0.50,0.50}{##1}}}
\expandafter\def\csname PY@tok@cp\endcsname{\def\PY@tc##1{\textcolor[rgb]{0.74,0.48,0.00}{##1}}}
\expandafter\def\csname PY@tok@k\endcsname{\let\PY@bf=\textbf\def\PY@tc##1{\textcolor[rgb]{0.00,0.50,0.00}{##1}}}
\expandafter\def\csname PY@tok@kp\endcsname{\def\PY@tc##1{\textcolor[rgb]{0.00,0.50,0.00}{##1}}}
\expandafter\def\csname PY@tok@kt\endcsname{\def\PY@tc##1{\textcolor[rgb]{0.69,0.00,0.25}{##1}}}
\expandafter\def\csname PY@tok@o\endcsname{\def\PY@tc##1{\textcolor[rgb]{0.40,0.40,0.40}{##1}}}
\expandafter\def\csname PY@tok@ow\endcsname{\let\PY@bf=\textbf\def\PY@tc##1{\textcolor[rgb]{0.67,0.13,1.00}{##1}}}
\expandafter\def\csname PY@tok@nb\endcsname{\def\PY@tc##1{\textcolor[rgb]{0.00,0.50,0.00}{##1}}}
\expandafter\def\csname PY@tok@nf\endcsname{\def\PY@tc##1{\textcolor[rgb]{0.00,0.00,1.00}{##1}}}
\expandafter\def\csname PY@tok@nc\endcsname{\let\PY@bf=\textbf\def\PY@tc##1{\textcolor[rgb]{0.00,0.00,1.00}{##1}}}
\expandafter\def\csname PY@tok@nn\endcsname{\let\PY@bf=\textbf\def\PY@tc##1{\textcolor[rgb]{0.00,0.00,1.00}{##1}}}
\expandafter\def\csname PY@tok@ne\endcsname{\let\PY@bf=\textbf\def\PY@tc##1{\textcolor[rgb]{0.82,0.25,0.23}{##1}}}
\expandafter\def\csname PY@tok@nv\endcsname{\def\PY@tc##1{\textcolor[rgb]{0.10,0.09,0.49}{##1}}}
\expandafter\def\csname PY@tok@no\endcsname{\def\PY@tc##1{\textcolor[rgb]{0.53,0.00,0.00}{##1}}}
\expandafter\def\csname PY@tok@nl\endcsname{\def\PY@tc##1{\textcolor[rgb]{0.63,0.63,0.00}{##1}}}
\expandafter\def\csname PY@tok@ni\endcsname{\let\PY@bf=\textbf\def\PY@tc##1{\textcolor[rgb]{0.60,0.60,0.60}{##1}}}
\expandafter\def\csname PY@tok@na\endcsname{\def\PY@tc##1{\textcolor[rgb]{0.49,0.56,0.16}{##1}}}
\expandafter\def\csname PY@tok@nt\endcsname{\let\PY@bf=\textbf\def\PY@tc##1{\textcolor[rgb]{0.00,0.50,0.00}{##1}}}
\expandafter\def\csname PY@tok@nd\endcsname{\def\PY@tc##1{\textcolor[rgb]{0.67,0.13,1.00}{##1}}}
\expandafter\def\csname PY@tok@s\endcsname{\def\PY@tc##1{\textcolor[rgb]{0.73,0.13,0.13}{##1}}}
\expandafter\def\csname PY@tok@sd\endcsname{\let\PY@it=\textit\def\PY@tc##1{\textcolor[rgb]{0.73,0.13,0.13}{##1}}}
\expandafter\def\csname PY@tok@si\endcsname{\let\PY@bf=\textbf\def\PY@tc##1{\textcolor[rgb]{0.73,0.40,0.53}{##1}}}
\expandafter\def\csname PY@tok@se\endcsname{\let\PY@bf=\textbf\def\PY@tc##1{\textcolor[rgb]{0.73,0.40,0.13}{##1}}}
\expandafter\def\csname PY@tok@sr\endcsname{\def\PY@tc##1{\textcolor[rgb]{0.73,0.40,0.53}{##1}}}
\expandafter\def\csname PY@tok@ss\endcsname{\def\PY@tc##1{\textcolor[rgb]{0.10,0.09,0.49}{##1}}}
\expandafter\def\csname PY@tok@sx\endcsname{\def\PY@tc##1{\textcolor[rgb]{0.00,0.50,0.00}{##1}}}
\expandafter\def\csname PY@tok@m\endcsname{\def\PY@tc##1{\textcolor[rgb]{0.40,0.40,0.40}{##1}}}
\expandafter\def\csname PY@tok@gh\endcsname{\let\PY@bf=\textbf\def\PY@tc##1{\textcolor[rgb]{0.00,0.00,0.50}{##1}}}
\expandafter\def\csname PY@tok@gu\endcsname{\let\PY@bf=\textbf\def\PY@tc##1{\textcolor[rgb]{0.50,0.00,0.50}{##1}}}
\expandafter\def\csname PY@tok@gd\endcsname{\def\PY@tc##1{\textcolor[rgb]{0.63,0.00,0.00}{##1}}}
\expandafter\def\csname PY@tok@gi\endcsname{\def\PY@tc##1{\textcolor[rgb]{0.00,0.63,0.00}{##1}}}
\expandafter\def\csname PY@tok@gr\endcsname{\def\PY@tc##1{\textcolor[rgb]{1.00,0.00,0.00}{##1}}}
\expandafter\def\csname PY@tok@ge\endcsname{\let\PY@it=\textit}
\expandafter\def\csname PY@tok@gs\endcsname{\let\PY@bf=\textbf}
\expandafter\def\csname PY@tok@gp\endcsname{\let\PY@bf=\textbf\def\PY@tc##1{\textcolor[rgb]{0.00,0.00,0.50}{##1}}}
\expandafter\def\csname PY@tok@go\endcsname{\def\PY@tc##1{\textcolor[rgb]{0.53,0.53,0.53}{##1}}}
\expandafter\def\csname PY@tok@gt\endcsname{\def\PY@tc##1{\textcolor[rgb]{0.00,0.27,0.87}{##1}}}
\expandafter\def\csname PY@tok@err\endcsname{\def\PY@bc##1{\setlength{\fboxsep}{0pt}\fcolorbox[rgb]{1.00,0.00,0.00}{1,1,1}{\strut ##1}}}
\expandafter\def\csname PY@tok@kc\endcsname{\let\PY@bf=\textbf\def\PY@tc##1{\textcolor[rgb]{0.00,0.50,0.00}{##1}}}
\expandafter\def\csname PY@tok@kd\endcsname{\let\PY@bf=\textbf\def\PY@tc##1{\textcolor[rgb]{0.00,0.50,0.00}{##1}}}
\expandafter\def\csname PY@tok@kn\endcsname{\let\PY@bf=\textbf\def\PY@tc##1{\textcolor[rgb]{0.00,0.50,0.00}{##1}}}
\expandafter\def\csname PY@tok@kr\endcsname{\let\PY@bf=\textbf\def\PY@tc##1{\textcolor[rgb]{0.00,0.50,0.00}{##1}}}
\expandafter\def\csname PY@tok@bp\endcsname{\def\PY@tc##1{\textcolor[rgb]{0.00,0.50,0.00}{##1}}}
\expandafter\def\csname PY@tok@fm\endcsname{\def\PY@tc##1{\textcolor[rgb]{0.00,0.00,1.00}{##1}}}
\expandafter\def\csname PY@tok@vc\endcsname{\def\PY@tc##1{\textcolor[rgb]{0.10,0.09,0.49}{##1}}}
\expandafter\def\csname PY@tok@vg\endcsname{\def\PY@tc##1{\textcolor[rgb]{0.10,0.09,0.49}{##1}}}
\expandafter\def\csname PY@tok@vi\endcsname{\def\PY@tc##1{\textcolor[rgb]{0.10,0.09,0.49}{##1}}}
\expandafter\def\csname PY@tok@vm\endcsname{\def\PY@tc##1{\textcolor[rgb]{0.10,0.09,0.49}{##1}}}
\expandafter\def\csname PY@tok@sa\endcsname{\def\PY@tc##1{\textcolor[rgb]{0.73,0.13,0.13}{##1}}}
\expandafter\def\csname PY@tok@sb\endcsname{\def\PY@tc##1{\textcolor[rgb]{0.73,0.13,0.13}{##1}}}
\expandafter\def\csname PY@tok@sc\endcsname{\def\PY@tc##1{\textcolor[rgb]{0.73,0.13,0.13}{##1}}}
\expandafter\def\csname PY@tok@dl\endcsname{\def\PY@tc##1{\textcolor[rgb]{0.73,0.13,0.13}{##1}}}
\expandafter\def\csname PY@tok@s2\endcsname{\def\PY@tc##1{\textcolor[rgb]{0.73,0.13,0.13}{##1}}}
\expandafter\def\csname PY@tok@sh\endcsname{\def\PY@tc##1{\textcolor[rgb]{0.73,0.13,0.13}{##1}}}
\expandafter\def\csname PY@tok@s1\endcsname{\def\PY@tc##1{\textcolor[rgb]{0.73,0.13,0.13}{##1}}}
\expandafter\def\csname PY@tok@mb\endcsname{\def\PY@tc##1{\textcolor[rgb]{0.40,0.40,0.40}{##1}}}
\expandafter\def\csname PY@tok@mf\endcsname{\def\PY@tc##1{\textcolor[rgb]{0.40,0.40,0.40}{##1}}}
\expandafter\def\csname PY@tok@mh\endcsname{\def\PY@tc##1{\textcolor[rgb]{0.40,0.40,0.40}{##1}}}
\expandafter\def\csname PY@tok@mi\endcsname{\def\PY@tc##1{\textcolor[rgb]{0.40,0.40,0.40}{##1}}}
\expandafter\def\csname PY@tok@il\endcsname{\def\PY@tc##1{\textcolor[rgb]{0.40,0.40,0.40}{##1}}}
\expandafter\def\csname PY@tok@mo\endcsname{\def\PY@tc##1{\textcolor[rgb]{0.40,0.40,0.40}{##1}}}
\expandafter\def\csname PY@tok@ch\endcsname{\let\PY@it=\textit\def\PY@tc##1{\textcolor[rgb]{0.25,0.50,0.50}{##1}}}
\expandafter\def\csname PY@tok@cm\endcsname{\let\PY@it=\textit\def\PY@tc##1{\textcolor[rgb]{0.25,0.50,0.50}{##1}}}
\expandafter\def\csname PY@tok@cpf\endcsname{\let\PY@it=\textit\def\PY@tc##1{\textcolor[rgb]{0.25,0.50,0.50}{##1}}}
\expandafter\def\csname PY@tok@c1\endcsname{\let\PY@it=\textit\def\PY@tc##1{\textcolor[rgb]{0.25,0.50,0.50}{##1}}}
\expandafter\def\csname PY@tok@cs\endcsname{\let\PY@it=\textit\def\PY@tc##1{\textcolor[rgb]{0.25,0.50,0.50}{##1}}}

\def\PYZbs{\char`\\}
\def\PYZus{\char`\_}
\def\PYZob{\char`\{}
\def\PYZcb{\char`\}}
\def\PYZca{\char`\^}
\def\PYZam{\char`\&}
\def\PYZlt{\char`\<}
\def\PYZgt{\char`\>}
\def\PYZsh{\char`\#}
\def\PYZpc{\char`\%}
\def\PYZdl{\char`\$}
\def\PYZhy{\char`\-}
\def\PYZsq{\char`\'}
\def\PYZdq{\char`\"}
\def\PYZti{\char`\~}
% for compatibility with earlier versions
\def\PYZat{@}
\def\PYZlb{[}
\def\PYZrb{]}
\makeatother


    % Exact colors from NB
    \definecolor{incolor}{rgb}{0.0, 0.0, 0.5}
    \definecolor{outcolor}{rgb}{0.545, 0.0, 0.0}



    
    % Prevent overflowing lines due to hard-to-break entities
    \sloppy 
    % Setup hyperref package
    \hypersetup{
      breaklinks=true,  % so long urls are correctly broken across lines
      colorlinks=true,
      urlcolor=urlcolor,
      linkcolor=linkcolor,
      citecolor=citecolor,
      }
    % Slightly bigger margins than the latex defaults
    
    \geometry{verbose,tmargin=1in,bmargin=1in,lmargin=1in,rmargin=1in}
    
    

    \begin{document}
    
    
    \maketitle
    
    

    
    \begin{Verbatim}[commandchars=\\\{\}]
{\color{incolor}In [{\color{incolor} }]:} \PY{o}{!}ls \PYZhy{}R
\end{Verbatim}


    \subsection{深度学习程序架构}\label{ux6df1ux5ea6ux5b66ux4e60ux7a0bux5e8fux67b6ux6784}

在做深度学习实验或项目时,为了得到最优的模型结果,中间往往需要很多次的尝试和修改。而合理的文件组织结构,以及一些小技巧可以极大地提高代码的易读易用性。根据我的个人经验,在从事大多数深度学习研究时,程序都需要实现以下几个功能:

\begin{itemize}
\tightlist
\item
  模型定义
\item
  数据处理和加载
\item
  训练模型(Train\&Validate)
\item
  训练过程的可视化
\item
  测试(Test/Inference)
\end{itemize}

    前面提到过,程序主要包含以下功能:

\begin{itemize}
\tightlist
\item
  模型定义
\item
  数据加载
\item
  训练和测试
\end{itemize}

首先来看程序文件的组织结构:

\begin{verbatim}
├── checkpoints/
├── data/
│   ├── __init__.py
│   ├── dataset.py
│   └── get_data.sh
├── models/
│   ├── __init__.py
│   ├── AlexNet.py
│   ├── BasicModule.py
│   └── ResNet34.py
└── utils/
│   ├── __init__.py
│   └── visualize.py
├── config.py
├── main.py
├── requirements.txt
├── README.md
\end{verbatim}

其中:

\begin{itemize}
\tightlist
\item
  \texttt{checkpoints/}:
  用于保存训练好的模型,可使程序在异常退出后仍能重新载入模型,恢复训练
\item
  \texttt{data/}:数据相关操作,包括数据预处理、dataset实现等
\item
  \texttt{models/}:模型定义,可以有多个模型,例如上面的AlexNet和ResNet34,一个模型对应一个文件
\item
  \texttt{utils/}:可能用到的工具函数,在本次实验中主要是封装了可视化工具
\item
  \texttt{config.py}:配置文件,所有可配置的变量都集中在此,并提供默认值
\item
  \texttt{main.py}:主文件,训练和测试程序的入口,可通过不同的命令来指定不同的操作和参数
\item
  \texttt{requirements.txt}:程序依赖的第三方库
\item
  \texttt{README.md}:提供程序的必要说明
\end{itemize}

    \subsubsection{关于\_\_init\_\_.py}\label{ux5173ux4e8e__init__.py}

可以看到,几乎每个文件夹下都有\texttt{\_\_init\_\_.py},一个目录如果包含了\texttt{\_\_init\_\_.py}
文件,那么它就变成了一个包(package)。\texttt{\_\_init\_\_.py}可以为空,也可以定义包的属性和方法,但其必须存在,其它程序才能从这个目录中导入相应的模块或函数。例如在\texttt{data/}文件夹下有\texttt{\_\_init\_\_.py},则在\texttt{main.py}
中就可以\texttt{from\ data.dataset\ import\ DogCat}。而如果在\texttt{\_\_init\_\_.py}中写入\texttt{from\ .dataset\ import\ DogCat},则在main.py中就可以直接写为:\texttt{from\ data\ import\ DogCat},或者\texttt{import\ data;\ dataset\ =\ data.DogCat},相比于\texttt{from\ data.dataset\ import\ DogCat}更加便捷。

    \subsubsection{数据加载}\label{ux6570ux636eux52a0ux8f7d}

数据的相关处理主要保存在\texttt{data/dataset.py}中。关于数据加载的相关操作,在上一章中我们已经提到过,其基本原理就是使用\texttt{Dataset}提供数据集的封装,再使用\texttt{Dataloader}实现数据并行加载。Kaggle提供的数据包括训练集和测试集,而我们在实际使用中,还需专门从训练集中取出一部分作为验证集。对于这三类数据集,其相应操作也不太一样,而如果专门写三个\texttt{Dataset},则稍显复杂和冗余,因此这里通过加一些判断来区分。对于训练集,我们希望做一些数据增强处理,如随机裁剪、随机翻转、加噪声等,而验证集和测试集则不需要。

    数据加载的一些常用库

\texttt{python\ import\ os\ from\ PIL\ import\ Image\ from\ torch.utils\ import\ data\ import\ numpy\ as\ np\ from\ torchvision\ import\ transforms\ as\ T}

    \begin{Shaded}
\begin{Highlighting}[]
\ImportTok{import}\NormalTok{ os}
\ImportTok{from}\NormalTok{ PIL }\ImportTok{import}\NormalTok{ Image}
\ImportTok{from}\NormalTok{ torch.utils }\ImportTok{import}\NormalTok{ data}
\ImportTok{import}\NormalTok{ numpy }\ImportTok{as}\NormalTok{ np}
\ImportTok{from}\NormalTok{ torchvision }\ImportTok{import}\NormalTok{ transforms }\ImportTok{as}\NormalTok{ T}

\KeywordTok{class}\NormalTok{ DogCat(data.Dataset):}
    
    \KeywordTok{def} \FunctionTok{__init__}\NormalTok{(}\VariableTok{self}\NormalTok{, root, transforms}\OperatorTok{=}\VariableTok{None}\NormalTok{, train}\OperatorTok{=}\VariableTok{True}\NormalTok{, test}\OperatorTok{=}\VariableTok{False}\NormalTok{):}
        \CommentTok{'''}
\CommentTok{        目标:获取所有图片地址,并根据训练、验证(train=True or False)、测试(test=True or False )划分数据}
\CommentTok{        '''}
        \VariableTok{self}\NormalTok{.test }\OperatorTok{=}\NormalTok{ test}
\NormalTok{        imgs }\OperatorTok{=}\NormalTok{ [os.path.join(root, img) }\ControlFlowTok{for}\NormalTok{ img }\KeywordTok{in}\NormalTok{ os.listdir(root)] }\CommentTok{#获取root路径下所有文件或文件夹的名称并将其与root组合起来作为该文件的绝对路径存放                                                  #到imgs列表里}

        \CommentTok{# test1: data/test1/8973.jpg}
        \CommentTok{# train: data/train/cat.10004.jpg }
        \ControlFlowTok{if} \VariableTok{self}\NormalTok{.test:}
\NormalTok{            imgs }\OperatorTok{=} \BuiltInTok{sorted}\NormalTok{(imgs, key}\OperatorTok{=}\KeywordTok{lambda}\NormalTok{ x: }\BuiltInTok{int}\NormalTok{(x.split(}\StringTok{'.'}\NormalTok{)[}\OperatorTok{-}\DecValTok{2}\NormalTok{].split(}\StringTok{'/'}\NormalTok{)[}\OperatorTok{-}\DecValTok{1}\NormalTok{]))}\CommentTok{#应该是把x先分割开,再按分割后的数字排序,key是用来进行比较的元素}
        \ControlFlowTok{else}\NormalTok{:}
\NormalTok{            imgs }\OperatorTok{=} \BuiltInTok{sorted}\NormalTok{(imgs, key}\OperatorTok{=}\KeywordTok{lambda}\NormalTok{ x: }\BuiltInTok{int}\NormalTok{(x.split(}\StringTok{'.'}\NormalTok{)[}\OperatorTok{-}\DecValTok{2}\NormalTok{]))}
                \CommentTok{#train文件的格文件名:'/home/lrl/pytorchlearning/pytorch-book/chapter6-实战指南/data/train/cat.8.jpg'}
                \CommentTok{#所以实际上split('.')将文件名分成了3部分,而split('.')[-2]就是读取中间那两个.之间的数字,并按这个数字排序}
                \CommentTok{#而test数据集采用split('.')[-2].split('/')[-1]先按/分割去最后一部分再按.取倒数第二部分}
\NormalTok{        imgs_num }\OperatorTok{=} \BuiltInTok{len}\NormalTok{(imgs)}\CommentTok{#获取元素个数}
        
        \CommentTok{# 划分训练、验证集,验证:训练 = 3:7}
        \ControlFlowTok{if} \VariableTok{self}\NormalTok{.test:}
            \VariableTok{self}\NormalTok{.imgs }\OperatorTok{=}\NormalTok{ imgs}
        \ControlFlowTok{elif}\NormalTok{ train:}
            \VariableTok{self}\NormalTok{.imgs }\OperatorTok{=}\NormalTok{ imgs[:}\BuiltInTok{int}\NormalTok{(}\FloatTok{0.7}\OperatorTok{*}\NormalTok{imgs_num)]}
        \ControlFlowTok{else}\NormalTok{ :}
            \VariableTok{self}\NormalTok{.imgs }\OperatorTok{=}\NormalTok{ imgs[}\BuiltInTok{int}\NormalTok{(}\FloatTok{0.7}\OperatorTok{*}\NormalTok{imgs_num):]            }
    
        \ControlFlowTok{if}\NormalTok{ transforms }\KeywordTok{is} \VariableTok{None}\NormalTok{:}
        
            \CommentTok{# 数据转换操作,测试验证和训练的数据转换有所区别}
            
\NormalTok{            normalize }\OperatorTok{=}\NormalTok{ T.Normalize(mean }\OperatorTok{=}\NormalTok{ [}\FloatTok{0.485}\NormalTok{, }\FloatTok{0.456}\NormalTok{, }\FloatTok{0.406}\NormalTok{], }
\NormalTok{                                     std }\OperatorTok{=}\NormalTok{ [}\FloatTok{0.229}\NormalTok{, }\FloatTok{0.224}\NormalTok{, }\FloatTok{0.225}\NormalTok{])}

            \CommentTok{# 测试集和验证集}
            \ControlFlowTok{if} \VariableTok{self}\NormalTok{.test }\KeywordTok{or} \KeywordTok{not}\NormalTok{ train: }
                \VariableTok{self}\NormalTok{.transforms }\OperatorTok{=}\NormalTok{ T.Compose([}
\NormalTok{                    T.Scale(}\DecValTok{224}\NormalTok{),}
\NormalTok{                    T.CenterCrop(}\DecValTok{224}\NormalTok{),}
\NormalTok{                    T.ToTensor(),}
\NormalTok{                    normalize}
\NormalTok{                ]) }
            \CommentTok{# 训练集}
            \ControlFlowTok{else}\NormalTok{ :}
                \VariableTok{self}\NormalTok{.transforms }\OperatorTok{=}\NormalTok{ T.Compose([}
\NormalTok{                    T.Scale(}\DecValTok{256}\NormalTok{),}
\NormalTok{                    T.RandomSizedCrop(}\DecValTok{224}\NormalTok{),}
\NormalTok{                    T.RandomHorizontalFlip(),}
\NormalTok{                    T.ToTensor(),}
\NormalTok{                    normalize}
\NormalTok{                ]) }
                
        
    \KeywordTok{def} \FunctionTok{__getitem__}\NormalTok{(}\VariableTok{self}\NormalTok{, index):}
        \CommentTok{'''}
\CommentTok{        返回一张图片的数据}
\CommentTok{        对于测试集,没有label,返回图片id,如1000.jpg返回1000}
\CommentTok{        '''}
\NormalTok{        img_path }\OperatorTok{=} \VariableTok{self}\NormalTok{.imgs[index]}
        \ControlFlowTok{if} \VariableTok{self}\NormalTok{.test: }
\NormalTok{             label }\OperatorTok{=} \BuiltInTok{int}\NormalTok{(}\VariableTok{self}\NormalTok{.imgs[index].split(}\StringTok{'.'}\NormalTok{)[}\OperatorTok{-}\DecValTok{2}\NormalTok{].split(}\StringTok{'/'}\NormalTok{)[}\OperatorTok{-}\DecValTok{1}\NormalTok{])}
        \ControlFlowTok{else}\NormalTok{: }
\NormalTok{             label }\OperatorTok{=} \DecValTok{1} \ControlFlowTok{if} \StringTok{'dog'} \KeywordTok{in}\NormalTok{ img_path.split(}\StringTok{'/'}\NormalTok{)[}\OperatorTok{-}\DecValTok{1}\NormalTok{] }\ControlFlowTok{else} \DecValTok{0}
\NormalTok{        data }\OperatorTok{=}\NormalTok{ Image.}\BuiltInTok{open}\NormalTok{(img_path)}
\NormalTok{        data }\OperatorTok{=} \VariableTok{self}\NormalTok{.transforms(data)}
        \ControlFlowTok{return}\NormalTok{ data, label}
    
    \KeywordTok{def} \FunctionTok{__len__}\NormalTok{(}\VariableTok{self}\NormalTok{):}
        \CommentTok{'''}
\CommentTok{        返回数据集中所有图片的个数}
\CommentTok{        '''}
        \ControlFlowTok{return} \BuiltInTok{len}\NormalTok{(}\VariableTok{self}\NormalTok{.imgs)}
\end{Highlighting}
\end{Shaded}

    \subsubsection{数据加载部分步骤的总结}\label{ux6570ux636eux52a0ux8f7dux90e8ux5206ux6b65ux9aa4ux7684ux603bux7ed3}

\begin{itemize}
\tightlist
\item
  加载常用库,设置读取图片的类,继承于torch.utils.data.Dataset,主要参数有,数据集路径,判断是训练验证还是测试集,是否进行数据格式转换,
\item
  构造函数部分,读入图片存储地址并排序后存放到一个列表里(按什么排序?数据没下完之前也不知道)采用spilt函数分割文件名
\item
  划分训练,验证,测试集
\item
  设置训练集与验证测试集不同的数据转换方式
\item
  设置\texttt{\_\_getitem\_\_()}返回一张图片的数据:读取列表中存储的地址,分别取出data与label,转换data数据格式,返回data,label
\item
  设置\texttt{\_\_len\_\_()},返回数据集中所有图片的个数
\end{itemize}

    采用\texttt{\_\_getitem\_\_()}加载图片的原因:

DataLoader里面并没有太多的魔法方法,它封装了Python的标准库\texttt{multiprocessing},使其能够实现多进程加速。在此提几点关于Dataset和DataLoader使用方面的建议:
1. 高负载的操作放在\texttt{\_\_getitem\_\_}中,如加载图片等。 2.
dataset中应尽量只包含只读对象,避免修改任何可变对象,利用多线程进行操作。

第一点是因为多进程会并行的调用\texttt{\_\_getitem\_\_}函数,将负载高的放在\texttt{\_\_getitem\_\_}函数中能够实现并行加速。
第二点是因为dataloader使用多进程加载,如果在\texttt{Dataset}实现中使用了可变对象,可能会有意想不到的冲突。在多线程/多进程中,修改一个可变对象,需要加锁,但是dataloader的设计使得其很难加锁(在实际使用中也应尽量避免锁的存在),因此最好避免在dataset中修改可变对象。例如下面就是一个不好的例子,在多进程处理中\texttt{self.num}可能与预期不符,这种问题不会报错,因此难以发现。如果一定要修改可变对象,建议使用Python标准库\texttt{Queue}中的相关数据结构。


    % Add a bibliography block to the postdoc
    
    
    
    \end{document}
